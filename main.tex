\documentclass[10pt,a4paper]{article}

\usepackage{titlesec}
\usepackage[round]{natbib}
\usepackage{authblk}
\usepackage{fullpage}
\usepackage[space]{grffile}
\usepackage{graphicx}
\usepackage{textcomp}
\usepackage{deluxetable}
\usepackage{longtable}
\usepackage[utf8]{inputenc}
\usepackage[ngerman,english]{babel}
\usepackage{float}
\usepackage[margin=1in]{geometry}
\usepackage{parskip}


%\usepackage{setspace}
%\usepackage[section]{placeins}
%\usepackage{xcolor}
%\usepackage{breakcites}
%\usepackage{lineno}
%\usepackage{hyphenat}
%\usepackage{latexsym}
%\usepackage{tabulary}
%\usepackage{booktabs,array,multirow}
%\usepackage{amsfonts,amsmath,amssymb}


\renewcommand{\familydefault}{\sfdefault}


\PassOptionsToPackage{hyphens}{url}
\usepackage[colorlinks = true,
linkcolor = blue,
urlcolor = blue,
citecolor = blue,
anchorcolor = blue]{hyperref}
\usepackage{etoolbox}
\makeatletter
%\patchcmd\@combinedblfloats{\box\@outputbox}{\unvbox\@outputbox}{}{%
% \errmessage{\noexpand\@combinedblfloats could not be patched}%
%}%
\makeatother



\renewenvironment{abstract}
 {{\bfseries\noindent{\abstractname}\par\nobreak}\footnotesize}
 {\bigskip}


% You can conditionalize code for latexml or normal latex using this.
\newif\iflatexml\latexmlfalse
\providecommand{\tightlist}{\setlength{\itemsep}{0pt}\setlength{\parskip}{0pt}}%

\AtBeginDocument{\DeclareGraphicsExtensions{.pdf,.PDF,.eps,.EPS,.png,.PNG,.tif,.TIF,.jpg,.JPG,.jpeg,.JPEG}}



% Edit this header.tex file to include frontmatter definitions and global macros
\newcommand{\beginappendix}{%
	\setcounter{table}{0}
	\renewcommand{\thetable}{A\arabic{table}}%
	\setcounter{figure}{0}
	\renewcommand{\thefigure}{A\arabic{figure}}%
}

% Add here any LaTeX packages you would like to load in all document blocks


% Add here any LaTeX macros you would like to load in all document blocks
% \def\example{This is an example macro.}

% -----

\iflatexml
% Add here any LaTeXML-specific commands

% -----

\else
% Add here any export style-specific LaTeX commands. These will only be loaded upon document export. 
% \paperfield{Subject domain of my document}
% \keywords{keyword1, keyword2}
% \corraddress{Author One PhD, Department, Institution, City, State or Province, Postal Code, Country}
% \fundinginfo{Funder One, Funder One Department, Grant/Award Number: 123456.}
\fi


\begin{document}

\title{Why do we have so many different hydrological models? A review based on the case of Switzerland}


\author[1]{Pascal Horton*}
\author[1]{Bettina Schaefli}
\author[1]{Martina Kauzlaric}
\affil[1]{Institute of Geography \& Oeschger Centre for Climate Change Research, University of Bern, Bern, Switzerland (pascal.horton@giub.unibe.ch)}


 \date{}


\begingroup
%\let\center\flushleft
%\let\endcenter\endflushleft
\maketitle
\endgroup


%This is a preprint of a manuscript submitted to WIREs Water.
%\newpage


\begin{abstract}
Hydrology plays a central role in applied as well as fundamental environmental sciences, but it is well known to suffer from an overwhelming diversity of models, in particular to simulate streamflow. Based on Switzerland's example, we discuss here in detail how such diversity did arise even at the scale of such a small country. The case study's relevance stems from the fact that Switzerland shows a relatively high density of academic and research institutes active in the field of hydrology, which led to an evolution of hydrological models that stands exemplarily for the diversification that arose at a larger scale. Our analysis summarizes the main driving forces behind this evolution, discusses drawbacks and advantages of model diversity and depicts possible future evolutions. Although convenience seems to be the main driver so far, we see potential change in the future with the advent of facilitated collaboration through open sourcing and code sharing platforms. We anticipate that this review, in particular, helps researchers from other fields to understand better why hydrologists have so many different models.
\end{abstract}%


\section{Introduction}
\label{sec:intro}

Hydrological models are essential tools for hydrologists, be it for
operational flood forecasting, water resource management or the
assessment of land use and climate change impacts. Since the advent of
hydrological modelling, the number of models keeps increasing at a fast
pace. It has become common to talk about the ``plethora of hydrological
models'' (index term found more than 13'400 times in a Google search on
11 Jan 2021). Single models are branching out into numerous variants,
such as the Hydrologiska Byråns Vattenbalansavdelning model
(HBV; \citealp{Bergstrom1976a,Bergstrom1992,Bergstrom1995,Lindstr_m_1997}) that exists in multiple versions nowadays.
Some authors support the idea that there are too many hydrological
models, which might lead to a waste of time and effort, and that the
hydrological community should gather on a Community Hydrological
Model \citep{Weiler_2015}.

While any newcomer to hydrological modelling will easily find some
guidance on navigating the sheer diversity of hydrological models,
understanding the concepts and limitations \citep{Beven_2013,Solomatine_2011,Kauffeldt_2016}, the
question of how this diversity has emerged receives much less attention.
Existing historical analyses of model diversity \citep{Peel_2020}
generally focus on the technical evolution of model types. According to
our personal experience, much of the knowledge about why many similar
models have emerged is transferred informally.

One of the key drivers for the pronounced model diversity in hydrology
is certainly the wide range of model
applications \citep{Weiler_2015} that all require \textit{appropriate
modelling}; this concept can be defined following \citet{Rosbjerg2005} as
``the development or selection of a model with a degree of
sophistication that reflects the actual needs for modelling results''.
Two well-accepted characteristics that models should exhibit are
parsimony and adequacy to the problem at hand, i.e. a model should not
be more complex than necessary and should be
fit-for-purpose \citep{Beven_2013}. Indeed, a model developed for
droughts cannot be blindly applied to the assessment of floods. Also,
catchments with different properties or climatology may require
different model structures \citep{Kavetski_2011,van_Esse_2013}. In other words, the
hydrological model diversification is strongly driven by the modelling
context and by what is now often called \textit{uniqueness of
place} \citep{Beven_2000}.

However, the hydrologic literature also offers other explanations,
ranging from legacy reasons for model selection \citep{Addor_2019}, to
a lack of agreement on concepts for process representations and to
the simple wish to try to do better with yet another model
parameterization \citep{Weiler_2015}.

We attempt here an analysis of what might explain the emergence of
multiple hydrological models at a rather small scale, the scale of
Switzerland, a country small enough to do an exhaustive analysis, but
diverse enough to shed light on some of the most dominant drivers of
model diversity. Despite Switzerland's small area (41285
km\textsuperscript{2}), numerous models are being developed and applied
in the same contexts and often even for the same purpose and the same
catchment. 

Thus, this work aims to disentangle the motivations and reasons behind
the choices that led to the current co-existence of a wide range of
models. We focus this analysis on hydrological models (see Box 1) that
simulate hydrological processes, including surface and subsurface flow,
and the resulting streamflow at the catchment scale. Some of these
models are classical rainfall-runoff models (Box 1), while others have
more specific purposes. We exclude here models that simulate the water
balance without providing streamflow at the catchment outlet. We first
briefly present the different models used in Switzerland
(Section \ref{sec:models}), and attempt a classification
according to types of application and research fields (Section
\ref{sec:application}), before presenting a synthesis of our
findings on drivers of model diversity
(Section \ref{sec:motivations}) and conclusions (Section \ref{sec:conclusion}).


\begin{quote}
\subsection*{Box 1: What do we mean by hydrological model ?}
\label{box:1}

A hydrological model is an input-output model that simulates the
evolution of water storage, of water fluxes and potentially of
associated chemical and physical properties at the Earth's surface and
subsurface, based on the water balance equation. The term
``rainfall-runoff model'' is often used for hydrologic models that
simulate streamflow at a catchment outlet based on input time series of
rainfall. The term ``rainfall-runoff'' stems from the early times when
such models simulated how much water of a rainfall event ran off to the
stream (rather than being stored in the catchment), i.e. ``runoff''
designated the part of rainfall that appears as
streamflow \citep{organization1992}. Nowadays, rainfall-runoff models are
continuous simulation tools that simulate all components of streamflow
(including baseflow), and the term ``runoff'' now designates the lateral
(as opposed to vertical) movement of water (at the surface or in the
subsurface) towards a river \citep{organization2012}. Modern rainfall-runoff
models further transform simulated hillslope-scale runoff to
catchment-scale streamflow; some of them include instream routing. Such
models can be generalized to precipitation-runoff models in the presence
of snowfall. The term ``water balance model'' is sometimes used as
synonym for rainfall-runoff models \citep{Boughton_2004} . The correcter
term ``rainfall-streamflow'' model appeared rather
early \citep{Young_1991} but is to date (12 Jan 2021) only used in 17
WebOfScience publications. Streamflow is in many papers called
interchangeably ``discharge'' and sometimes even ``runoff'', which is a
legacy effect. 
\end{quote}


\section{Hydrological models developed and used in Switzerland}
\label{sec:models}

\subsection{Preliminary remark}
\label{sec:models:remark}

The information sources considered in this analysis are as far as
possible peer-reviewed articles with applications to hydrology. The
articles were retrieved based on searches by authors (hydrologists in
Switzerland) and keywords. While we tried to search all applications as
exhaustively as possible, biases in the search and citing network
effects are possible if not likely. Where necessary, conference
proceedings, PhD theses, research and government reports are also
included. A few models are exclusively used or developed in engineering
companies, and these are not included here. Furthermore, our analysis
focuses on catchment-scale modelling and excludes studies that focus on
hydrogeological modelling \citep{Carlier2019} and those with a focus on
urban hydrology \citep{Peleg2017} or urban
hydrogeology \citep{schirmer2013}. All articles are not directly
referenced in this paper, but a complete table is available in the
supplementary material.

\subsection{Model overview}
\label{sec:models:overview}

There are several hydrological models that have been developed in
Switzerland (Table \ref{tab:modelslist}), ranging from
rainfall-runoff models (PREVAH, GSM-SOCONT, RS, SEHR-ECHO, WaSiM-ETH),
to snow-based models (ALPINE3D), glacier-hydrology models (GERM) and
water temperature models (StreamFlow). Some models
(Table \ref{tab:modelslist}) have their roots outside
Switzerland but are now actively being developed in Switzerland
(HBV-light, TOPKAPI-ETH, SUPERFLEX) or were applied to Swiss case
studies (CemaNeige-GR6J, LARSIM, VIC, SWAT, mHM). 

All these models are briefly described in Appendix 1. Switzerland being
an Alpine country, most of these models include a representation of snow
accumulation and melt, some also include glacier-related processes.


\hspace{-4cm}\begin{deluxetable}{l l l c}
	\vskip4mm
	\centering
	\tabletypesize{\footnotesize}
	\tablecolumns{4} 
	\tablecaption{List of models (alphabetical order) applied in Switzerland; the fourth column indicates whether the model was originally developed (D) or further evolved (E) by teams active at Swiss universities or research institutes, or whether it is only applied to Swiss case studies, either by teams active in Switzerland (A-CH) or by teams active abroad (A). References are in the main text and Appendix 1. \label{tab:modelslist}}
	\tablewidth{0pt}
	\tablehead{
		\colhead{Model name} & \colhead{Full name} & \colhead{Spatial structure} & \colhead{Type of use}}
	\startdata 
	ALPINE3D & ALPINE3D & distributed & D  \\
	CemaNeige-GR6J & CemaNeige - Genie Rural \`{a} 6 param\`{e}tres Journalier & lumped & A-CH \\
	DECIPHeR & Dynamic fluxEs and ConnectIvity for Predictions of HydRology & HRU-based & E \\
	GERM & Glacier Evolution Runoff Mode & distributed & D \\
	GSM-SOCONT  & Glacier and SnowMelt {SOil CONTribution model} & semi-distributed & D  \\
	HBV & Hydrologiska Byråns Vattenbalansavdelning & semi-distributed & A\\
	HBV-light & Hydrologiska Byråns Vattenbalansavdelning - light & semi-distributed &  E  \\
	HYPE  & HYdrological Predictions for the Environment & semi-distributed &  A  \\
	LISFLOOD & LISFLOOD & distributed &  A \\
	LARSIM & Large Area Runoff Simulation Model & semi-distributed &  A \\
	mHM & meso-scale hydrological model & distributed &  A \\
	PREVAH & Precipitation-Runoff-Evapotranspiration HRU Model & HRU-based \& distributed &  D  \\
	RS & Routing System & semi-distributed &  D  \\
	SEHR-ECHO  & Spatially Explicit Hydro. Response model for ecohydro. applic. & semi-distributed &  D  \\
	StreamFlow & StreamFlow & distributed &  D  \\
	SUPERFLEX & SUPERFLEX & (not fixed) &  E  \\
	SWAT  & Soil Water and Assessment Tool & semi-distributed &  A-CH,A \\
	TOPKAPI-ETH & TOPographic Kinematic APproximation and Integration - ETH & distributed &  E  \\
	VIC & Variable Infiltration Capacity model & distributed &  A \\
	WaSiM-ETH & Water Flow and Balance Simulation Model - ETH & distributed &  D  \\
	wflow & wflow & distributed &  A  \\
	\enddata
\end{deluxetable}


\subsection{History of hydrological modelling in Switzerland}
\label{sec:models:history}

The early times of hydrological modelling in Switzerland can be situated
in the years 1970 to 1990, when model diversity naturally emerged in
response to modelling needs. From a hydrological processes perspective,
a strong focus was on the simulation of snowmelt
runoff \citep{braun1986} as well as on understanding the role of
forests in the water cycle \citep{keller1991,forster1989}. Along with modelling
studies in experimental catchments \citep{Iorgulescu1994}, first model-based
climate change \citep{bultot1992a} and land-use change \citep{jordan1990a}
impact studies appeared. Quantitative real-time forecasts for water
resources management \citep{p1969} and hydropower
production \citep{jensenlang1973} started being based on hydrologic models
rather than statistical approaches.

It is worth noting that \citet{naef1977} presented already a first
model intercomparison study, comparing complex and simple models. In
fact, model diversity already started interpellating the research
community in the late 1970ties and \citet{f1981} notably asked:
``But, given that the results are good, why do new models continue to be
published?'' 

From a historical analysis (see Supplementary Information), one
interesting aspect can be retained: already in the early times of model
development, part of the model diversity resulted from the work of
geoscientists and engineers not directly specialized in catchment
hydrology \citep{Abednego1990,k1986a,hager1984,sautier1980}, which partly explains the parallel
emergence of similar models. 

Additional details on the emergence of hydrological modelling in
Switzerland are given in the Supporting Information. For a more general
dive into the history of modelling, the reader is referred to the work
of Keith Beven \citep{Beven_2020,Beven_2020a}.

\subsection{Model intercomparison}
\label{sec:models:intercomparison}

There are few model intercomparison studies in Switzerland, and none of
them led to a preference of one model over the other, which is reflected
in the fact that the recent re-evaluation of climate change impact on
water resources involved all the major models developed in Switzerland
 \citep{bafu2021}. The existing model intercomparison studies are
classical studies on what we can gain from model complexity in terms of
model performance; e.g. \citet{Gurtz2003} compared PREVAH to
WaSiM-ETH, \citet{Orth2015} HBV-light to PREVAH and a simple water
balance model, \citet{Kobierska_2013} ALPINE3D to PREVAH
and \citet{Andrianaki2019} SWAT to ALPINE3D and PREVAH.


\section{Fields of model application and drivers of diversity}
\label{sec:application}

To structure the analysis of model diversity, we attempt here a
clustering of modelling studies according to the underlying application.
We propose a focus on the following categories of applications or
research areas that have received signification attention in Switzerland
(Fig. \ref{fig:map}): hydrologic process research
(Section \ref{sec:application:process}), real-time forecasting 
(Section \ref{sec:application:forecasting}), characterization and 
quantification of floods (Section \ref{sec:application:floodsdroughts}), 
climate change impact analysis (Section \ref{sec:application:climatechange}), 
ecohydrology and agricultural water use (Section \ref{sec:application:echohydrology}),
sediment production and transport (Section \ref{sec:application:sediment}), analysis of model
behaviour and uncertainty analysis (Section \ref{sec:application:uncertainty}) and large scale
modelling (Section \ref{sec:application:largescale}).

\begin{figure}[htb]
	\begin{center}
		\includegraphics[width=0.95\columnwidth]{figures/Map}
		\caption{{Map of Switzerland with its major drainage divides (green), the extent
				of the ``hydrological Switzerland'' (orange) and some catchments (brown)
				that are referenced in the text (Data: Federal Office of Topography
				swisstopo and Hydrological Atlas of Switzerland). 
				\label{fig:map}
		}}
	\end{center}
\end{figure}

\subsection{Hydrologic process research}
\label{sec:application:process}

Very few studies use catchment-scale hydrological models to assist
hydrological process research and hypothesis testing within a hydrologic
model development framework \citep{clarkOpinion2016}. This might be explained
by the fact that it remains highly challenging to draw conclusions on
hydrological processes based on model simulations at the catchment
scale; corresponding work rather involves small scale modelling at the
hillslope scale (e.g. the study of \citealt{Heuvel2018} from the US). 

One example is the work of \citet{Comola2017} that analyzes how solar
radiation patterns influence the snow-hydrologic response based on two
models of different complexity (ALPINE3D and SEHR-ECHO). ALPINE3D was
also used by \citet{Hindshaw2011} to attribute the origin of systematic
seasonal and diurnal variations in glacial stream water chemistry, and
by \citet{Brauchli2017} to assess the influence of small-scale snowmelt
variations on the catchment-scale hydrologic response.

Another example of model-assisted process research is the analysis
of \citet{Paschalis_2014} with TOPKAPI-ETH on the interplay of rainfall's
temporal variability and the clustering of saturated areas in flood
generation.

Although there are currently few studies of this type, this topic holds
potential for significant model diversification since hydrologic process
analysis might require model structural changes to allow new hypotheses
to be tested. An example is given by \citet{DalMolin2020}, who discusses,
based on the SUPERFLEX framework, how to flexibly adapt the model
structure to integrate new hypotheses about dominant hydrological
processes.

\subsection{Real-time forecasting}
\label{sec:application:forecasting}

The ever increasing need for reliable real-time streamflow forecasts
leads to a continuous evolution of the underlying hydro-meteorological
modelling systems. Real-time forecasting started with deterministic
forecasts from a single meteorological forecast applied to a single
hydrological model; today, users expect full stochastic ensemble
forecasts at hourly time scales, updated every few hours and with
several stochastic meteorological inputs applied to different
hydrological models \citep{Karsten2016}. Coupled atmospheric--hydrologic
ensemble prediction systems were proven to provide better forecasts
than deterministic simulations \citep{Verbunt2007,Zappa2008,Jaun2008a,Liechti2013}. These might also
include data assimilation schemes \citep{J_rg_Hess_2015} or the assessment
of hydrologic uncertainty related to meteorological forcings, model
parameters and initial conditions \citep{Zappa2011a,Fundel2011}.

Such modern forecasting systems require hydrological models that provide
forecasts at many locations in a stream network, that are fast to run,
and that include the effect of hydraulic infrastructures (eg. of
hydropower water intakes and accumulation lakes). Since the early times
of flood forecasting, HBV and PREVAH were used in governmental offices
\citep{Karsten2016} as well as in research institutes because of their
relative simplicity and low computational costs \citep{Verbunt2006,Addor_2011,Murphy_2019,Antonetti2019}.

PREVAH also plays a prominent role in drought
forecasting \citep{Fundel2013,J_rg_Hess_2015,Bogner2018b} and within the operational Swiss drought
information platform (\citealp{Stahli2013}). Furthermore, it is to date
the only model used for subseasonal streamflow
forecasts \citep{Monhart_2019,Anghileri2019}, which is still in its infancy in
Switzerland.

Despite the dominance of HBV and PREVAH, the considerably more complex
WaSiM-ETH model has, however, also been used for research studies on
improving flood forecasting in mountainous areas \citep{Jasper2003,Ahrens2003b,Jasper2002}.
Along with HBV, PREVAH and LARSIM, WaSim-ETH is today part of the Swiss
operational ensemble forecasting system \citep{Karsten2016}, which uses
the FEWS platform (Flood Early Warning System; \citealp{Werner_2013}) to
provide forecasts for the cantonal authorities and the
public \citep{FOEN2019}. A key advantage of the computationally
intensive WaSiM-ETH model is the fact that it can explicitly account for
lake regulations and hydropower operations (J. Schulla, personal
communication, October 23, 2020). 

In parallel to the above-mentioned models, RS MINERVE is being used as a
specific flood forecasting tool for the upper Rhone river catchment, a
large catchment (5220 km\textsuperscript{2}, see
Fig. \ref{fig:map}) strongly influenced by glacier
melt and hydropower production \citep{GarciaHernandez2009b,GarciaHernandez2009,Jordan2010}. Before the recent
implementation in WaSiM-ETH and in TOPKAPI-ETH, RS MINERVE was the only
operational tool that explicitly modelled the effect of lakes and
hydraulic infrastructures.

\subsection{Characterization and quantification of floods and droughts}
\label{sec:application:floodsdroughts}

Infrastructure planning, water resources and natural risk management
also heavily rely on probabilistic quantifications of extremes, i.e. an
estimation of what could happen in terms of floods and droughts and
their associated probabilities (called return periods in hydrology).
Work in this field continues to be based on statistical analyses and
extrapolation of observed streamflow time series \citep{Brunner2018,Asadi20108}, but
hydrological models play an ever-increasing role to complement missing
or insufficient streamflow data.

Any model-based flood estimation method is computationally intensive
since long model simulation runs are required at an hourly time step.
Accordingly, simple models such as PREVAH \citep{Viviroli_2009,Viviroli2009c,Felder2017},
HBV-light \citep{Brunner_2019a,Sikorska2017,Sikorska-Senoner2020} and 
RS \citep{Zeimetz2018,Zeimetz2017,Bieri2013} dominate the
Swiss literature on flood estimation; these models are all deemed to
perform well enough for flood estimation in Swiss catchments by their
respective authors and users. 

However, given that all the above simple models rely on similar
reservoir-based streamflow simulation methods, there is currently an
important modelling effort by Kauzlaric (personal communication) to
diversify flood estimation modelling for flood risk assessment, through
the further development of the modular and open-source model DECIPHeR
for Swiss catchments.

Other complex distributed models are to date only used to study specific
flood types that involve a good physical parameterization of small scale
processes, such as rain-on-snow flood events, as in the study
by \citet{Rossler2014} with WaSiM-ETH. 

In addition to the above studies, there are also applications aiming at
the reconstruction and/or reproduction of historical floods dating
further in the past. The effect of a major volcanic eruption in 1816 on
the generation of floods in the upper Rhine basin (see
Fig. \ref{fig:map}) has been analysed
by \citet{Rossler2018} using WaSiM-ETH. \citet{Stucki2018} reconstructed
a large flood of the 19th Century in Ticino with PREVAH coupled with the
routing part of RS (the model was chosen because it was already
calibrated for the region
by \citet{Andres_2016}). \citet{Ancey2019} reconstructed the 1818
Giétro glacial lake outburst flood with GERM (which was also ready to
use for this area). These examples show that models developed for
current day conditions are transposed without further adaptation to
historical conditions, similarly as they are transposed to future
conditions (see the following section).

Work on droughts is much less abundant in Switzerland than work on
floods, which is related to the fact that missing water was, in the
past, not a hot topic in this country known as the water tower of
Europe \citep{Milano2015}. What can be highlighted here is that the same
models are in use to assess droughts and floods, potentially with
specific recalibration, but without modifying the model structure. This
is motivated by the fact that existing models are deemed to reproduce
well all dominant processes in the Swiss environment, as e.g. explicitly
stated in the work of \citet{Zappa2007a} on quantifying the hydrological
impact of the 2003 heatwave with a distributed version of PREVAH, later
on used for additional drought analyses \citep{Brunner2019,Zappa2019}. Similarly,
HBV-light served in several drought studies \citep{Staudinger2014,Staudinger2014a,Staudinger2015} and was
used to assess low flow drivers in Alpine catchments \citep{Arnoux_2020}.

However, significant efforts to improve the model representation of
groundwater and the corresponding baseflow during droughts remain to be
done in Switzerland, which will most probably lead to further model
diversification.

\subsection{Climate change impact analysis}
\label{sec:application:climatechange}

Climate change impact studies emerged in Switzerland in the 1990s,
including a large national research programme on climate change and
natural hazards \citep{snfs}. Since then, all model-based studies
are mostly conducted with the models that established themselves in
Switzerland, which have, however, not been specifically designed for
climate change impact analysis; detailed assessments of how well these
models can simulate future conditions are largely missing.

WaSiM-ETH was, for example, chosen by \citet{Jasper2004} to assess the
effect of different regional climate scenarios in the Thur and the
Ticino catchments (Fig. 1). The model choice is justified by the fact
that ``from the hydrological point of view of spatially distributed
catchment modelling, this model represents the state-of-the-art'' and
that it ``can be successfully applied to a wide range of scales''. 
WaSiM-ETH has also been applied to assess future soil water
patterns \citep{Jasper2006,Rossler2012} and future summer evapotranspiration
regimes \citep{Calanca2006}. It was even applied for the entire Rhine
basin at a 1 km\textsuperscript{2} resolution down to Rotterdam
by \citet{Kleinn_2005}.

TOPKAPI-ETH, the other frequently used complex distributed model has
also been used in several climate change impact
applications \citep{Fatichi2014,Fatichi2015,Finger_2012,Anghileri2018}. 

{The most widely used models to study climate change impact on
streamflow are to date however the reservoir-based models PREVAH
(}\citealt{Koplin2012,Bosshard2013a,Speich_2015,Junker_2014} and others; see Supplementary material) and
HBV-light (\citealt{Etter2017,Hakala2020,Brunner_2018,Jenicek2018} and others). In the western part of
Switzerland, RS and GSM-SOCONT were used in the past, especially for
high elevation sites \citep{Horton2006,Uhlmann_2012,Uhlmann2013a,Terrier2015}.

The justification of these models for climate change impact assessments
is well summarized by \citet{K_plin_2010} who, for PREVAH, states that
the model ``has been developed especially to suit conditions in
mountainous environments'' and that it ``has proved to be a reliable and
flexible tool for various scopes of application and climate conditions
ranging from drought analysis over water balance modelling to flood
estimation and forecasting''. 

It is in the context of climate change impact studies that we see for
the first time the use of an internationally well-established model, the
SWAT model, with an application to the Upper Rhone river
catchment \citep{Rahman2014}. SWAT was not specifically designed for
Alpine environments, but it provides the interesting possibility to
study the impact of vegetation-related land-use
changes \citep{Rahman2015}. Later on, SWAT was also applied
by \citet{Zarrineh2020} to an agricultural region in Western Switzerland
to assess the impact of climate change on streamflow, erosion, and
agriculture.

Overall, studies on the impact of vegetation changes remain extremely
rare in Switzerland; examples include the analysis of forest change
by \citet{zierl05} with the ecohydrologic model RHESSys (not further
used in Switzerland), by \citet{Koplin2013} and \citet{Schattan2013} with
PREVAH and by \citet{Alaoui2014} with WaSiM-ETH. The work
of \citet{Milano2015a} with PREVAH is to date the only study accounting
also for anthropogenic effects on future water stress. 

There is, however, an ample body of literature on the study of glacier
retreat impacts on hydrology, which can be seen as a land-use change
effect \citep{Horton2006,Schaefli2007b,Finger2015,Etter2017,Addor2014,Junghans2011}. 
This namely gave rise to the development of
the GERM model \citep{Huss2016,Junghans2011,Farinotti2012,Finger2013} 
and a new glacier retreat parameterization scheme widely applied
internationally \citep{Huss2010}.

The question of how to model the effect of warming on snow accumulation
and melt deserves special attention. Most hydrological models used in
Switzerland rely on a simple temperature-index based snow routine. A
more complex snow routine has been recently implemented in 
WaSiM-ETH \citep{Thornton2019}. ALPINE3D, which is built on the
physically-based SNOWPACK model (\citealt{Lehning2002,Bartelt_2002,Bartelt2002,Lehning_2002}), 
has certainly the most complex representation of snow processes among Swiss 
models and has been used in several climate change impact
applications \citep{Bavay2009,Bavay2013,Marty2017}. However, detailed comparisons between
simple snow routines and ALPINE3D have not been conclusive so far with
respect to climate change applications \citep{Kobierska2011,Shakoor2018}. This
long-standing question of how to model future snow will most likely see
additional model diversification in the future. 

In this context, we would like to stress here that there are still
relatively few examples of hydrologic model ensembles (using several
hydrologic models) for climate change impact
assessment \citep{Kobierska2011,Addor2014}. The expectations in this regard might
raise in the future, requiring the use of additional, and likely
internationally widely applied, hydrological models.

\subsection{Ecohydrology and agricultural water use}
\label{sec:application:echohydrology}

Ecohydrology studies the feedbacks between ecosystems, in particular
with vegetation and the water cycle \citep{tague200}. To date,
catchment-scale studies with ecohydrological models accounting for
feedback with vegetation are scarce in Switzerland. An early example is
a work of \citet{zierl05}, who used the model RHESSys to study
climate and land-use change impacts on alpine streamflow in
Switzerland. 

This topic will gain importance in the future, e.g. to study transport
phenomena of chemicals \citep{queloz2015}, nutrient and pollutant
cycling, C0\textsubscript{2} production or water
temperature \citep{michel2020}. In research, this might lead to
increased use of complex, distributed models that can be coupled to
ecosystem models, such a TOPKAPI-ETH \citep{Pappas_2015} or
STREAMFLOW \citep{Gallice_2016}.

Or, perhaps more likely, we will see the development of new models
specifically targeted at vegetation-hydrology interactions, such as the
one developed by \citet{Fatichi2012,Fatichi2012a} called Tethys-Chloris (T\&C) and
developed to simulate vegetation-hydrology interactions at large scales.
It has been applied to catchments in Switzerland to study soil moisture
spatiotemporal dynamics \citep{Fatichi2015a}, as well as to assess the
vulnerability of Alpine ecosystems to climate change \citep{Mastrotheodoros2019}.

Similarly, there are very few studies on agricultural water use, a topic
that will gain importance in the future, but that might not further
drive model diversification. For example, WaSiM-ETH has already been
shown to be suitable to study the demand and supply of water for
agriculture, including irrigation \citep{Fuhrer2012}. Besides,
internationally widely used models might see more applications in
Switzerland. One such example is SWAT (e.g. \citealt{Abbaspour2007} ), which
- given its user-friendliness - is likely to see more alpine
applications despite not being specifically targeted to alpine
environments \citep{Andrianaki2019}, but in exchange offers the option to
study land management effects \citep{Zarrineh2018}.

\subsection{Sediment production and transport}
\label{sec:application:sediment}

A special topic that deserves some focus is sediment transport
modelling. The modelling of sediment sources, as well as transport
capacity, requires models that yield reliable spatial patterns of
hydrological processes, i.e. complex distributed models such as
TOPKAPI-ETH, which is being extended for this
purpose \citep{Konz_2011,Battista2020}. Sediment management, for example in the
context of hydropower production \citep{RaymondPralong2015,Gabbud2015}, is constantly
gaining importance in Alpine countries and will most likely drive the
development of new modules to simulate the interplay of hydrological and
geomorphological processes at the catchment scale.

\subsection{Analysis of model behaviour and uncertainty analysis}
\label{sec:application:uncertainty}

A large body of hydrologic modelling literature focuses on a better
understanding of model behaviour and in particular of model performance
with respect to reproducing observed streamflow, e.g. as a function of
model parameterizations, of spatio-temporal model resolution
\citep{Brunner_2019}, of precipitation input data \citep{Sikorska2016,Muller-Thomy2019}, 
or of parameter estimation techniques \citep{Foglia_2009}. In the
context of model diversity, this field of research has overall little
impact because most modelling groups who work on such theoretical
aspects, simply use their in-house models for proofs of concepts or to
actually improve their them (e.g. \citealt{Schaefli2007,Hingray2010}).

The HBV-light model is probably the most widely used with this respect;
model performance studies range from the integration of glacier mass
balance data \citep{Finger2015,schaeflihuss11}, of snow data
assimilation \citep{Griessinger2016}, accounting for streamflow observation
uncertainty \citep{Westerberg2020}, the influence of spatial or temporal
resolution of hydro-meteorological input \citep{GironsLopez2016,Sikorska2018}, to the
integration of citizen science data \citep{Etter2020}.

Similarly, there are several studies on model calibration and
performance with PREVAH; examples include the study on error correction
in forecasting chains by \citet{Bogner_2018}, investigation of parameter
regionalisation with little observed streamflow data
by \citet{Viviroli2015} or the assessment of spatial pattern reproduction
of soil moisture and evapotranspiration \citep{Zappa2003} and of snow
\citep{Zappa2008a}.

More complex models are rarely used in this context due to computational
constraints. Examples include the work of \citet{Cullmann2011}, which
used WaSiM-ETH to compare the efficiency of different methods for
parameter estimation, or the work of \citet{Rossler2019}, which also
used WaSiM-ETH to compare various climate postprocessing methods and
quantify their impact on different hydrological signatures. 

Overall, there are only very few parameter regionalisation studies; one
example is the work of \citet{Melsen2016}, who applied the
internationally used VIC model to the Thur catchment to assess the
impact of parameter transfer over different temporal and spatial
resolutions.

\subsection{Large scale modelling}
\label{sec:application:largescale}

To complete the picture, we address here the application of some
international hydrological models implemented for Europe or large
European river basins such as the Rhine, and thus covering at least a
major part of the hydrological domain of Switzerland. However, we
restrict ourselves to those models whose code is publicly available
and/or whose results are published and/or directly available for Swiss
basins. 

\citet{Kauffeldt_2016} presented a technical review of large-scale
hydrological models implemented in operational forecasting schemes on a
continental level with regard to their suitability for the European
Flood Awareness System (EFAS). Amongst the models evaluated in the
study, three have been deployed specifically for Europe: LISFLOOD, HYPE
and mHM (see Appendix 1).

While LISFLOOD and HYPE are already running operationally (see Appendix
1) at the European scale, mHM has only recently been applied for the
development and evaluation of a pan-European multimodel seasonal
hydrological forecasting
system \citep{Wanders_2019}. \citet{Rottler2020} applied it for assessing
the potential future changes in flood seasonality in the Rhine River
with a 500m spatial resolution. 

Several other models have been applied specifically for the Rhine basin,
mainly focusing either on forecasting discharge or climate change impact
applications. Examples include the so-called wflow\_hbv model
(\citealp{van_Osnabrugge_2017}, \citealp{van_Osnabrugge_2019} and \citealt{van_Osnabrugge}) for
hourly/daily streamflow forecasting of the Rhine, allowing lake level
data assimilation. Another example is the LARSIM model, which was
implemented at a 1km\textsuperscript{2} resolution in combination with
HBV-light to assess the origin of streamflow components in a project of
the International Commission for the Hydrology of the Rhine basin
(KHR/CHR) in 2012 \citep{m2017}. The major regulated and
unregulated lakes were included in LARSIM, and also four of the most
influent ``clustered'' hydropower reservoirs present on the upper Aare,
upper Reuss, upper Rhine and in the Ill river catchment
(Fig. \ref{fig:map}).

In general, the skill of most large scale models is found to be inferior
near the main Alpine ridge compared to mountainous or lowland areas. The
high Alpine catchments have been identified early as posing a major
challenge to large scale hydrological modelling \citep{Kleinn_2005}. 
Besides larger errors in the meteorological variables (precipitation in
particular), and the important effect of water management practices, the
smaller the catchment area and the greater the elevation ranges, the
more detailed the model structure and the spatial resolution need to be
to achieve good model performances \citep{Gurtz2003}. While most likely
being cancelled out downstream \citep{Kleinn_2005}, these problems
remain yet to be addressed in large scale modelling and certainly partly
explain why specific Swiss-scale models continue to be extremely
popular. 

\section{Disentangling the motivations behind the choices}
\label{sec:motivations}

In total, we reviewed 157 peer-reviewed journal articles on hydrological
modelling in Swiss catchments (see Table 1 in the Supporting
Information). Excluding the large scale applications (Section \ref{sec:application:largescale}), a
Swiss hydrological model (category D or E in Table 1) is selected in
93\% of cases, leaving little room for international models. PREVAH
takes the lion's share with about 30\% of the applications, followed by
HBV-light (16.5\%) and WaSiM-ETH (14.6\%). The most used international
model is SWAT with a small 4\% usage (7 cases), mainly related to
research led from outside Switzerland. An analysis of the temporal
evolution of model use (Fig. \ref{fig:bars})
suggests a small opening to international models. The relatively higher
peak in 2003 is mainly due to articles related to the MAP project
(Mesoscale Alpine Programme).

\begin{figure}[htb]
	\begin{center}
		\includegraphics[width=0.70\columnwidth]{figures/Picture1}
		\caption{{{Number of articles} reporting applications of hydrological models over
				the years with a distinction of the models developed in Switzerland and
				international models.
				{\label{fig:bars}}
		}}
	\end{center}
\end{figure}

\citet{Addor_2019} argue that the choice of a model is driven by legacy
rather than adequacy, where they understand by legacy: ``practicality,
convenience, experience, and habit''. This hypothesis implies that the
hydrological model of choice depends on the one hand essentially on the
experience available in the modelling group and on the other hand on
the code and data availability. 

In the articles analyzed here, about 25\% of the authors specifically
address the model's adequacy with the context or the landscape. However,
this does not mean that adequacy has been formally tested or that it
actually drove the choice of the model, but rather that it is argued as
suitable to the intended application. About 53\% of the articles do not
provide any mention of adequacy. The rest provide some description of
the model characteristics that might be interpreted as arguments for
suitability to the case study. Furthermore, besides the hydrological
processes studies (Section \ref{sec:application:process}), there are no examples where the
perceptual model, i.e. the modellers' perception of how nature works, is
explicitly discussed, a shortcoming that is the rule rather than the
exception in hydrological modelling works \citep{beven2021}.

\begin{figure}[htb]
	\begin{center}
		\includegraphics[width=0.70\columnwidth]{figures/Rplot03}
		\caption{{Hydrological models applied to different contexts in Switzerland. The
				importance of the link is proportional to the number of scientific
				articles. The importance of some models can be inflated by the fact that
				an article can address multiple contexts, such as floods and climate
				change. Models with too few use cases (less than three) are not included
				for the sake of clarity. A3D stands for ALPINE3D and G-SCNT for
				GSM-SOCONT.
				{\label{fig:applications}}
		}}
	\end{center}
\end{figure}

Some models are specialized for certain processes, such as ALPINE3D for
snow and GERM for glaciers, and are thus proportionally more used in
these contexts (Fig. \ref{fig:applications}). TOPKAPI-ETH tends
to be more used for processes that need a physically-based
representation or that need a gridded spatial structure to provide
gridded output, e.g. in view of coupling to another model. RS
specifically targets flood modelling and hydropower operations, as it
was designed for operational flood mitigation with hydropower plants.
The three most used models, i.e. PREVAH, HBV-light and WaSiM-ETH, are
general models and are applied to different topics, such as climate
change impact studies, floods, droughts, cryosphere-related processes,
and operational forecasting (Fig. \ref{fig:applications}).

One point to note is that the adequacy to climate change studies is
generally not discussed. References to previous studies are sometimes
provided, without the latter having addressed this point explicitly.
While it is relatively easy to demonstrate a model's ability to
reproduce floods or drought conditions, its transferability to other
climate conditions is more difficult to prove directly.

One of the reasons driving the choice for one hydrological model,
besides the adequacy, is reusing a model that is already set up for a
catchment of interest, sometimes without the need to recalibrate it.
About 20\% of the articles explicitly state that the applied model comes
from another study. Likely, this number should be higher as some authors
publish multiple articles targeting different topics but on one model
setup, without explicitly mentioning it.

Another reason is the selection of the model that is developed and used
at the research institute. This is a strong driver, as for 66\% of the
articles the first author is affiliated with the institute where the
model is being developed - which confirms a hypothesis that is indeed
widespread in hydrology. The model at hand is thus well known, the code
is available, and specialists are present to make necessary adaptations
to it. Also, when the model is not one from the institute,
collaborations are established with the developer or the lead researcher
of the model, leading to the fact that for 72\% of the articles, the
model developer (or team leader) is co-authoring the paper. These three
aspects - reuse of an existing setup, in-house knowledge and
collaborations - have one common ground, which is convenience. Finally,
the choice of the model might also be imposed by a project.

While we see an impressive model diversity at the single or few
catchments scale, it is noteworthy that our review points towards an
important national-scale shortcoming: while there are catchments that
have been ``over-modelled'' (e.g. Thur, Dischma) - however with little
model intercomparison - there is a clear lack of large scale or national
studies. In particular, the few available studies often do not cross the
Swiss national border, even though the ``hydrological Switzerland''
extends to its neighbouring countries
(Fig. \ref{fig:map}).

The absence of such larger scale studies might be explained by
shortcomings and challenges more widely encountered in hydrological
modelling over larger domains. These include differing quality and
scales of input data and streamflow observations and large heterogeneity
in hydrological behaviour (possibly requiring more than one specialized
model). Yet, this heterogeneity may in fact provide us with the
opportunity to improve our understanding of differences in model
adequacy and model performance, and to draw most needed conclusions on
the robustness of generalizations and on estimation
uncertainty \citep{Gupta_2014,McMillan_2016}.


\section{Conclusion}
\label{sec:conclusion}

Focusing on Switzerland, we carried out a comprehensive literature
review on the hydrological models developed and applied in different
contexts. The objective of this work was to disentangle the motivations
and reasons behind the choices that led to the current co-existence of a
wide range of streamflow models in a small country. To structure the
analysis, we attempted a classification into eight fields of model
application, ranging from hydrological process research to model
uncertainty analysis and large scale modelling. For all reviewed
studies, we examined the arguments that were explicitly put forward by
the authors for model selection, as well as implicit aspects, such as
the author's affiliation or co-authorships. 

The model adequacy for the study context or the landscape is explicitly
addressed by only 25\% of the articles, while 53\% make no mention of
adequacy, neither provide any justification for the choice of the model.
The models that have specifically been developed to represent specific
processes, such as snow- or glacier melt or hydropower operations, are
obviously mainly used in these contexts. However, the more general
models, which are also the most used ones, are applied indifferently to
various contexts and landscapes.

Not surprisingly, researchers active in Switzerland are very keen on
using a model developed in Switzerland (93\% of the case studies) or
even at their own research institute (66\% of the articles analysed),
and possibly used on the same catchment previously (20\%), which all in
all underlines that convenience might be the foremost model selection
driver. Moreover, this is likely to be the cause of the existence of so
many hydrological models, as each research group develops its own tools.

Convenience certainly also explains that some catchments are used in
numerous studies and that larger scale or multi-catchment studies on
hydrological functioning and model behaviour are largely missing: both
points might in fact be explained by how tedious it remains to gather
all relevant data (Switzerland does not yet have a hydrological data
portal). The absence of model intercomparison, in exchange, might at
least partly be explained by the few open-source models used in
Switzerland. 

With ongoing climate change and ensuing challenges for water resources
and water-related hazard management, hydrological modelling needs to
evolve quickly. In Alpine environments, the most striking example is
certainly the emergence of hydrological droughts \citep{Van_Loon_2015}
during summer and fall \citep{brunner2019drought,Rigling2020}, which requires to understand
the drivers of low flow \citep{Arnoux_2020} and the development of
hydrological models that reliably represent groundwater recharge.

This component is, in fact, crudely parametrised in many streamflow
models for alpine environments. Improved modelling of surface
water-groundwater interactions is also a pre-condition for water
temperature projections, agricultural water use and related water
quality, drinking water management, and biodiversity assessment in
ecosystems strongly influenced by river-groundwater
interactions \citep{Brunner_2017}. 

Another key topic that will receive growing attention is the role of the
vegetation in modulating climate extremes \citep{Mastrotheodoros_2020} and
land-use changes induced by climate warming, calling thereby for
improved representations of vegetation's role in hydrological models.

Accordingly, application-oriented as well as essentially
research-oriented models can see further diversification in the near
future. If model development continues the path taken so far, models
will branch into sub-variants, and process-specific models will be
created. However, we see two elements that might reverse the trend. The
first element is the emergence of modular frameworks that allow creating
a wide variety of model structures. While some specific topics might
still need custom models tailored to certain applications, most
hydrological models share similar principles and process
representations. The creation of such flexible frameworks is strongly
encouraged by \citet{Clark2011a}. Nowadays, different flexible
frameworks exist, such as SUPERFLEX, FUSE \citep{Clark2008},
PERSiST \citep{Futter2014}, ECHSE \citep{Kneis2015},
MARRMoT \citep{Knoben2019}, Raven \citep{Craig2020}, and
SUMMA \citep{Clark2015}. {However, the flexibility provided by these
frameworks likely comes with a counterpart, which is more code
complexity. Most of these frameworks are relatively new and their
adoption by a large community of modellers remains to be proven}. 

The second element is the growing adoption of version control systems
that allow collaboration on open-source code with unprecedented ease.
These code sharing platforms (code repositories) allow for anyone to
suggest improvements (in a written form) to an open-source code or even
to suggest changes to the code (e.g. pull requests) that will be
reviewed by the developers of the model and merged to the main code
base. As more models go open source, the need to create in-house
versions to implement processes decreases. Hopefully, it should increase
contributions to shared code bases and benefit a community-driven
dynamic that would be beneficial for all, increasing thereby
international collaborations.


\section*{Acknowledgements}

We would like to thank various colleagues for their help on the history
of hydrological modelling in Switzerland (see Supplementary Information)
and Karsten Jasper for the insight on hydrological modelling at the
Swiss Federal Office for the Environment (FOEN).


\section*{Appendix 1: Short model descriptions, alphabetical order}
\label{appendix:1}

\textbf{ALPINE3D} \citep{Lehning_2006} is a model developed in Switzerland
targeting surface processes in alpine environments, in particular snow
processes, and is suitable for very steep terrain. It targets
applications where the small-scale variability at the atmosphere-surface
interface is important. Three-dimensional aspects relate to processes in
the atmosphere, such as drifting snow. The snow-related processes are
modelled by the physically-based SNOWPACK model (\citealt{Lehning2002,Bartelt_2002,Bartelt2002,Lehning_2002}).
ALPINE3D has a built-in runoff module adapted from an early version of
PREVAH \citep{Lehning_2006} and a runoff module that solves the Richards
equations \citep{Wever2017}. It has been recently extended by a
hydrological simulation tool for streamflow and water temperature
prediction \citep{Gallice_2016}.

\textbf{CemaNeige-GR6J} is the daily version of a lumped, bucket-type
rainfall-runoff model with six free parameters \citep{Pushpalatha2011},
combined with the CemaNeige snow module \citep{Valery2014a,Valery2014b}, which is a
routine for snow accumulation and melt based on a degree-day concept
that introduces two additional free parameters. GR6J is an empirical
model with a root zone storage and two routing routines: one for the
slow (unit hydrograph) and one for the fast flow component ({unit
hydrograph}, a non-linear and an exponential store). Both flow
components interact with the groundwater through an exchange
coefficient. It has seen one application in Switzerland for a climate
change impact study \citep{Keller2019a}.

\textbf{DECIPHeR} (Dynamic fluxEs and ConnectIvity for Predictions of
HydRology; \citealp{Coxon_2019}) is an open-source flexible model
framework suited for different spatial scales. The model builds on the
code and key concepts of Dynamic TOPMODEL (\citealp{Beven_2001}), an
improvement of the original TOPMODEL (TOPography based hydrological
model; \citealp{BEVEN_1979}). It can be run as a lumped model (1 HRU), as
semi-distributed (multiple HRUs) or as fully distributed (HRU for every
single grid cell). Each HRU is treated as a separate functional unit in
the model and thus allows for different process conceptualizations and
parameterizations across the catchment.

\textbf{GERM} (Glacier Evolution Runoff Model; \citealt{Huss2016,Farinotti2012})
consists of five different modules, which largely rely on existing
approaches, dealing with snow accumulation, ablation, glacier evolution,
evapotranspiration and runoff routing. It is a fully distributed,
deterministic, conceptual model designed mainly for simulations at a
daily resolution and a high spatial resolution. Glacier geometry is
updated annually according to a non-parametric approach proposed
by \citet{Huss2010}. The hydrological module is based on the concept
of linear reservoirs and distinguishes five surface types: ice, snow,
rock, vegetation and open water. 

\textbf{GSM-SOCONT} (Glacier and SnowMelt -- SOil CONTribution
model; \citealp{Schaefli2005c}) is a semi-lumped conceptual
glacio-hydrological model composed of the reservoir-based SOCONT model
(consisting in a linear reservoir for the slow soil contribution and a
non-linear reservoir for direct runoff) and the GSM model for the
glacierized area. the SOCONT model was inspired by the GR3
model \citep{Edijatno_1989}, which is part of the GR model family as is
CemaNeige-GR6J (see above). It was developed at the Ecole Polytechnique
Fédérale de Lausanne (EPFL). The model has a parsimonious structure and
was initially developed for climate change impact studies. Catchments
are subdivided first into ice-covered and ice-free parts and then in
elevation bands. A version of GSM-SOCONT has been implemented into RS
(see below) and modified for operational flood forecasting
\citep{Hamdi2005} and for design flood estimation \citep{Zeimetz_2018}.

\textbf{HBV} (Hydrologiska Byråns Vattenbalansavdelning
model; \citealp{Bergstrom1976a,Bergstrom1992,Bergstrom1995,Lindstr_m_1997}) is a rainfall-runoff model that focuses
on runoff generation processes, including snowm, and is characterized,
by its original developers \citep{Bergstrom1992}, as being very general and
is thus applied in many different geographical and climatological
conditions. 

\textbf{HBV-light} \citep{Seibert_2012} is an implementation of the HBV
model (see above) that is further developed at the University of
Zurich. HBV-light corresponds to a simplified and userfriendly version
of the original model. 

\textbf{HYPE} (HYdrological Predictions for the
Environment, \citealp{Lindstr_m_2010}) is a large-scale semi-distributed
conceptual model, designed to simulate discharge and model flow paths of
nutrients in the water, and was originally developed by the Swedish
Meteorological and Hydrological Institute. In the model, the landscape
is divided into classes according to the soil type, land use and
altitude, and the parameters are either global or coupled to the soil
type or land-use. The model can simulate natural hydrological processes
of snow- and glacier melt, evapotranspiration, soil moisture,
groundwater and routing through rivers and lakes, but also human-induced
influences, such as regulated lakes and reservoirs, water abstractions
and irrigation. HYPE is run operationally by SMHI for several purposes
(e.g. flood forecasting or climate change impact assessments). The
version covering the pan-European continent is referred to as E-HYPE,
its application is entirely based on open and readily available data
sources \citep{Donnelly_2015}, including historical data (1981-2010), 1-10
day forecast, seasonal forecasts, climate change impact scenarios and
actual model performance
(\url{https://hypeweb.smhi.se/explore-water/geographical-domains/\#europehype}).

\textbf{LARSIM} (Large Area Runoff Simulation Model; \citealp{Ludwig2006})
is a semi-distributed hydrological model, which describes continuous
runoff processes in catchments and river networks. The model structure
(subunit) can be grid-based or based on hydrologic subcatchments. While
runoff generation (described with parallel linear storage reservoirs),
routing (depending on channel geometries and roughness conditions) and
flow retention are simulated at the subunit scale, snow storage,
evapotranspiration, interception and soil storage are simulated at a
subscale level according to land use classes. While it doesn't include
a glacier melt component, LARSIM includes many features that were
specifically designed for its operational use as a flood forecasting
model, as well as offline applications \citep{m2017}. 

\textbf{LISFLOOD} is a GIS-based model for catchment-scale water balance
simulation \citep{vanderkniff2010}. It has been specifically designed for
large river catchments, and in particular, it makes use of data layers
that are available for the Joint Research Center (JRC) at European
scale, such as land use, soil type and texture, river
network \citep{Thielen_2009}. LISFLOOD is used by the European Flood
Awareness System, EFAS, for medium- and seasonal-range forecasts with a
6-hourly and daily time step. Both historical river discharge time
series (1991 to near real-time) and reforecasts (1999-2018) are
available on the Climate Data Store of Copernicus
(\url{https://cds.climate.copernicus.eu/}). 

\textbf{mHM} (mesoscale Hydrological model; \citealt{Samaniego2010a,Kumar_2013,Thober_2019}) is a
distributed hydrological model, which has the particularity of using the
multiscale parameter regionalization approach (MPR, \citealp{Samaniego2010a})
for parameter identification. It has been specifically developed to not
need recalibration when applied at different resolutions
\citep{Kauffeldt_2016}. It is driven by hourly or daily meteorological
forcings and utilizes observable basin physical characteristics to infer
the spatial variability of the required parameters. It is developed by
the Umweltforschungszentrum Leipzig and has been successfully applied to
catchments ranging from 4 km\textsuperscript{2} and to beyond 500,000
km\textsuperscript{2}. To the best of our knowledge, it does not yet
have a glacier melt component. The open-source code (Fortran) is
available at \url{https://git.ufz.de}.

\textbf{PREVAH} (Precipitation-Runoff-Evapotranspiration HRU
Model; \citealt{Gurtz1999,Viviroli_2009a}) is a Swiss conceptual model that has been
developed specifically for heterogeneous mountainous environments with
highly spatially and temporally variable processes. It follows the HBV
model structure, with numerous modifications, and was designed for
studies in Alpine headwater basins \citep{Orth2015}. PREVAH branched
out into different versions, two of which are mostly used: an HRU-based
version that runs at an hourly time step and a fully distributed version
\citep{Zappa2012} that runs at a daily time step. The distributed
version of PREVAH being the most used, any reference to PREVAH in this
paper implies the distributed version if not stated otherwise.

\textbf{RS} (Routing System; \citealp{Dubois2000,GarciaHernandez2007}) has been developed at the
Swiss Federal Institute of Technology in Lausanne (EPFL). The version
that is freely available, and thus more used, is RS MINERVE, which was
developed for operational flood forecasting in Valais \citep{GarciaHernandez2009b}
and which is maintained by the CREALP (Centre de recherche sur
l'environnement alpin). RS specifically targets hydropower systems by
modelling the influence of regulated infrastructures and thus allows
modelling complex hydrological and hydraulic networks with anthropogenic
influences.

\textbf{SEHR-ECHO} (Spatially Explicit Hydrologic Response model for
ecohydrologic applications; \citealp{Schaefli2014}) is an evolution of
GSM-SOCONT that was developed at the Swiss Federal Institute of
Technology in Lausanne (EPFL). The model aims at taking into account the
spatial variability in the runoff generation. The catchment is divided
into subcatchments connected to the river network in order to account
for the origin of the different areal contributions and to route them in
the river network. 

\textbf{StreamFlow} is an extension of ALPINE3D (see above). It uses an
explicit formulation of travel times \citep{Comola2015}, as does
SEHR-ECHO (see above). 

\textbf{SUPERFLEX} \citep{Fenicia2011a,Kavetski2011} is a flexible hydrological framework
now developed at Eawag (the Swiss Federal Institute of Aquatic Science
and Technology). It allows building hydrological models using generic
components for hypothesis testing. The building blocks are reservoirs,
lag functions, and connections. The models elaborated with SUPERFLEX can
be lumped, semi-distributed \citep{Fenicia2016} or fully
distributed \citep{Hostache2020}. An open-source version written in Python
is available \citep{Molin2020}.

\textbf{SWAT} (Soil Water and Assessment Tool; \citealp{Arnold_1998}) is an
open-source semi-distributed, process-based hydrological model. Besides
hydrology, other SWAT components can simulate energy balance, soil
temperature, mass transport and land management at the sub-basin and HRU
levels. It is one of the most applied models worldwide probably because
of the broad range of hydrologic and environmental problems that can be
addressed with it.

\textbf{TOPKAPI-ETH} \citep{Finger_2011,Ragettli_2012}, developed at ETH Zurich, is a
branch of the TOPKAPI model (TOPographic Kinematic APproximation and
Integration; \citealp{Todini1995,Todini2002,Liu2002,Ciarapica_2002}). It is a fully distributed and
physically-based model based on the spatial integration of the kinematic
wave model over the pixels of the digital elevation model (DEM).
TOPKAPI-ETH has been modified for application to mountain basins by
adding a second soil layer and modules for snow, glaciers, reservoirs,
water abstraction, and diversion, and a new evapotranspiration
scheme \citep{Finger_2011,Finger_2012,Fatichi_2015a}.

\textbf{VIC} (Variable Infiltration Capacity model; \citealp{Liang1994a}) is
an open-source grid-based land surface hydrological model. It is
implemented so that grid cells with a resolution up to 1km are simulated
independently of each other. Sub-grid heterogeneity introduced by
different land-use types and elevation is handled via statistical
distributions. Routing must be performed separately with an additional
routine taking care of the water transport between cells.

\textbf{WaSiM-ETH} (Water Flow and Balance Simulation
Model-ETH; \citealp{Schulla2007,Schulla2009}) is a fully distributed hydrological
model originally developed at ETH Zurich. It describes the water fluxes
in the unsaturated soil using the 1D-Richards
equation \citep{Richards1931}. The transfer function (runoff
concentration) can be processed through a series of linear reservoirs or
with the kinematic wave approach (from one cell to another). WaSiM-ETH
covers a wide range of hydrological processes relevant for alpine
environments, with different implemented variants.

\textbf{wflow} is the modular and distributed hydrological modelling
platform of DELTARES
(\url{https://www.deltares.nl/en/software/wflow-hydrology/}).
\textit{wflow\_hbv} is a fully distributed version of the conceptual HBV
model \citep{Lindstr_m_1997} - applied on a grid basis - in the wflow
framework with a kinematic wave as routing instead of the original
triangular routing function; the model has an interception reservoir,
snow module, root zone storage, fast runoff reservoir, and a groundwater
reservoir \citep{de_Boer_Euser_2017}. 


%\FloatBarrier
\bibliographystyle{apalike}
\bibliography{biblio}

\end{document}

